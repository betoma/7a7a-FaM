\documentclass[a4paper,10pt,twoside,openright]{memoir}
\usepackage{multicol, multirow, array}
\usepackage{fontspec}
\usepackage{anyfontsize}
\usepackage[pagecolor=none,dvipsnames]{xcolor}
\usepackage{ragged2e}
\usepackage{amsmath}
\usepackage{amssymb}
\usepackage[hidelinks]{hyperref}
\usepackage{url}
\usepackage[margin=0.8in]{geometry}
\usepackage{float, hhline}
\usepackage{booktabs}
\usepackage{textcomp}
\usepackage{expex}
\usepackage{enumitem}
\usepackage{paracol}
\usepackage[calc,english]{datetime2}
\usepackage{suffix}
\usepackage{afterpage}
\usepackage{phonrule}

%-----CONFIGURATION------
%------------------------

\setmainfont{Noto Serif}
% \setCJKmainfont[BoldFont=NotoSansCJKjp-Bold,AutoFakeSlant=0.15]{Noto Sans CJK JP}
% \setCJKsansfont[BoldFont=NotoSansCJKjp-Bold,AutoFakeSlant=0.15]{Noto Sans CJK JP}
\restylefloat{table}

\setsecnumdepth{subsubsection}
\settocdepth{subsubsection}

\definelingstyle{default}{glstyle=nlevel,numoffset=3em,textoffset=1.5em,exskip=.75ex,interpartskip=3ex,belowglpreambleskip=.25ex,aboveglftskip=.25ex,everyglft=\it}
\definelingstyle{Conversation}{aboveexskip=0pt,belowexskip=0pt,belowglpreambleskip=.5ex,aboveglftskip=0pt,glwordalign=left,glftpos=right,glhangstyle=none,
glrightskip=0pt plus .3\hsize,% added to avoid overfull glosses
glneveryline={\it\addfontfeatures{Letters=UppercaseSmallCaps}},%
ssratio=.5,% width of the left panel of the glwidth
everyglft={},%
}

\lingset{lingstyle=default}

\DTMnewdatestyle{eurodate}{%
    \renewcommand{\DTMdisplaydate}[4]{%
        \number##3.\nobreakspace%           day
        \DTMmonthname{##2}\nobreakspace%    month
        \number##1%                         year
    }%
    \renewcommand{\DTMDisplaydate}{\DTMdisplaydate}%
}

\DTMsetdatestyle{eurodate}

\renewcommand{\arraystretch}{1.4}

%-----description command---%
%---------------------------%

\SetLabelAlign{parrightcent}{\strut\smash{\parbox[c]{\labelwidth}{\raggedleft#1}}}
\SetLabelAlign{parright}{\strut\smash{\parbox[t]{\labelwidth}{\raggedleft#1}}}

%-------COMMANDS---------
%------------------------

\newcommand{\lang}{{\bigglot}a{\bigglot}a-\textsc{f}a\textsc{m}}
\newcommand{\longv}{ː}
\newcommand{\sqbrack}[1]{$\langle$#1$\rangle$}
\newcommand{\phipa}[1]{/#1/}
\newcommand{\bripa}[1]{[#1]}
\newcommand{\ttilde}{\raise.17ex\hbox{$\scriptstyle\sim$}}
\newcommand{\rootpart}{$\Theta$}
\newcommand{\glotstop}{ʔ}
\newcommand{\bigglot}{Ɂ}
\newcommand{\lilglot}{ɂ}
\newcommand{\nm}{\symbol{"2205}}
\newcommand{\tiebar}{͡}
\newcommand{\famwordold}[5]{#1\textsc{#2}#3\textsc{#4}#5}
\newcommand\famword[1]{{\addfontfeatures{Letters=UppercaseSmallCaps}#1}}
\newcommand{\famq}[1]{»#1«}

%----QoL COMMAND-------
%----------------------

\newcounter{numbertable}

\newcommand{\wtf}[1]{\thenumbertable. & #1 \\\refstepcounter{numbertable} }

%\newcommand{numbertablerow}[1]{%
%    \refstepcounter{numbertable}
%    \thenumbertable. & #1 }

%-----DICT COMMANDS------
%------------------------

\makeatletter
\@beginparpenalty=10000
\makeatother

\newcounter{dictwordcount}
\newcounter{definition}

\newenvironment{dictroot}[2]%
    {%
    \subsection{\uppercase{#1---#2}}
    \begin{description}[leftmargin=*,labelwidth=*]
    }{%
    \end{description}
    }%

\newcommand{\dictsubtitle}[1]{%
    \end{description}
    \subsubsection*{#1}
    \begin{description}[leftmargin=*,labelwidth=*]
}%

\makeatletter
\newenvironment{dictentry}[2]%
    {%
    \item[\famword{#1}] $\bullet$ \textit{#2}\hfill
    \protected@edef\@currentlabelname{#1}%
    \setcounter{definition}{0}%
    \refstepcounter{dictwordcount}%
    \begin{description}[align=right,labelwidth=*,font=\normalfont]
    }{%
    \end{description}
    }%
\makeatother

\newcommand{\dictdef}[1]{\refstepcounter{definition}%
\item[\thedefinition.] #1
}%

\WithSuffix\newcommand\dictdef*[1]{%
    \item[] #1
}

\newcommand{\newentry}[2]{%
\item[#1] $\bullet$ \textit{#2}\hfill
}%

%-------TITLE PAGE-------
%------------------------

\title{{\fontsize{100}{100}\selectfont \lang} \\ \Huge \sffamily A Reference Grammar of the Narish Language}
\author{Bethany E. Toma, Kim U. Korsæth}
\date{\today}

%--------MAIN DOC--------
%------------------------

\begin{document}

\pagecolor{Melon}
\maketitle
\pagecolor{white}

\frontmatter

\chapter{Foreword}

\lang{} is a constructed language spoken on the fictitious Nareland island.

\newpage

\tableofcontents

\mainmatter

\chapter{Phonology}
\section{Consonants}

\begin{table}[ht]
    \centering
    \begin{tabular}{rcccccc}
    \toprule
            & Labial & Alveolar & Palatal & Velar & Uvular & Glottal \\
    \midrule
    Fortis & pʰ \ttilde{} p\tiebar ɸ & tʰ \ttilde{} t\tiebar s &
    % \multirow{2}{*}{c \ttilde{} c\tiebar ç} & \multirow{2}{*}{k \ttilde{} k\tiebar x} & \multirow{2}{*}{q \ttilde{} q\tiebar χ} & \multirow{2}{*}{ʔ} \\
    c \ttilde{} c\tiebar ç & k \ttilde{} k\tiebar x & q \ttilde{} q\tiebar χ & \multirow{2}{*}{ʔ} \\
     Lenis & p \ttilde{} b & t \ttilde{} d & & & & \\
    Fricative & f & s & \multicolumn{4}{c}{ç \enspace \ttilde{} \enspace x \enspace \ttilde{} \enspace χ \enspace \ttilde{} \enspace ħ \enspace \ttilde{} \enspace h} \\
    Approximant & & l & j & w & & \\
    Nasal & m & n & & & & \\
    Rhotic & & \multicolumn{2}{c}{ɾ \enspace \ttilde{} \enspace ɹ \enspace \ttilde{} \enspace ɻ } & & & \\
    \bottomrule
    \end{tabular}
    \caption{Phonemic Consonant Inventory}
    \label{tab:consinv}
\end{table}

\lang[] distinguishes 6 places of articulationa and 5-6 manners of articulation.

Plosives are separated into fortis and lenis, where fortis plosives are unvoiced and strongly aspirated or affricated, and lenis are unaspirated and often voiced.

Three fricative phonemes have been identified, one labial, one coronal, and one dorsal. The dorsal fricative can be realized as anywhere from glottal to palatal.

The approximants /j w/ tend to be syllabic when not adjacent to a vowel, realized as [ɪ ʊ]. This process is related to that of epenthetic schwa insertion and occurs in the same environments. 
% Instead of inserting an epenthetic schwa and having the approximants glide off to that vowel quality, the schwa merges with the approximant to create `colored' reduced monophthongs. 

\begin{table}[ht]
    \centering
    \begin{tabular}{lll}
        \famword{JaMJ} `season' & /jamj/ & [ˈjamɪ] \\
        \famword{KLaTW} `swamp, bog' & /kʰlatʰw/ & [ˈkʰlatʰʊ] \\
        \famword{WSiiN} `bring about' & /wsi\longv{}n/ & [ʊˈsi\longv{}n] \\
        \famword{JPaNia} `Japan' & /jpʰania/ & [ɪˈpʰania] 
    \end{tabular}
\end{table}

\section{Vowels}

\begin{table}[ht]
    \centering
    \begin{tabular}{rccc}
    \toprule
          & Front & Central & Back \\
    \midrule
    Close & i (i\longv{}) & & u (u\longv) \\
    Mid   & & ə & \\
    Open  & & a (a\longv) & \\
    \bottomrule
    \end{tabular}
    \caption{Phonemic Vowel Inventory}
    \label{tab:vowelinv}
\end{table}

\section{Phonotactics \& Allophony}

Syllables always contain a vowel nucleus, with rarely more than two onset consonants and two coda consonants. Sonority hierarchy plays a big role in the structure of syllables and their realization.

\subsection{Sonority Hierarchy}
\subsection{High vowel lowering}

The high vowels /i i\longv{} u u\longv/ are lowered to [e e\longv{} o o\longv{}] before /j w/ respectively.

\subsection{Epenthetic schwa}

The epenthetic schwa appears within consonant sequences that, without adjacent vowels, violate syllable structure and may not be realized. It occurs in the earliest non-initial position possible that provides legal consonant sequences. This schwa is never stressed. For example, the word \famword{NurRKi} is phonemically /ˈnurrkʰi/, but \lang{} does not allow duplicate continuant phonemes, leaving the second /r/ out of the first syllable. The following syllable would not be able to accommodate it either, as it would violate the sonority rules by placing a less sonorous phoneme between two more sonorous ones. Of the two positions for the epenthetic schwa within this sequence, [ˈnuɾəɾkʰi] and [ˈnuɾɾəkʰi], only the former solution accommodates both restrictions and yields a legal realization.

\begin{table}[ht]
    \begin{tabular}{lll}
        \textit{Example}    & \textit{Phonemic transcription}   & \textit{Phonetic Realization} \\
        \famword{NurRKi}    & /ˈnurrkʰi/                        & [ˈnuɾəɾkʰi] \\
        \famword{mBiT}      & /ˈmpitʰ/                          & [məˈbitʰ] \\
        \famword{inNiM}     & /inˈnim/                          & [inəˈnim] \\
        \famword{SaFR}      & /ˈsafr/                           & [ˈsafər] 
    \end{tabular}
\end{table}

Word-onset sequences may never be disambiguated by prefixing an epenthetic schwa to the word. Like in medial and coda sequences, the epenthetic schwa must occur interconsonantally, but not across word boundaries. In the sequence |...C$_a$\#C$_x$C$_y$V...|, where C$_a$ is a consonant in the preceding word and C$_x$C$_y$ is an illegal sequence, the epenthetic schwa may not attempt to separate C$_a$ and C$_x$ in any way, and must instead separate C$_x$ and C$_y$. Word boundaries are inherently segmenting and preclude other segmenting elements like the epenthetic schwa. However, compound words are considered to be one phonological word, and in lacking a word boundary must make use of the epenthetic schwa.

\section{Prosody}

Stress, in the form of elevated pitch and volume, is placed on the first non-schwa vowel of the word, after the first root radical, on a long vowel immediately preceding the first radical, or on certain morphemes that carry stress.

\begin{table}[ht]
    %\centering
    \begin{tabular}{lll}
        nemiwi & [nəˈmiwi] & first non-schwa vowel of word\\
        parse & [ˈparsə] & first non-schwa vowel of word\\
        \famword{iFaaM} & [iˈfa{\longv}m] & vowel after first radical\\
        \famword{FanaS} & [ˈfanas] & vowel after first radical\\
        \famword{aaNiW}& [ˈa{\longv}niw] & long vowel preceding radical\\
        \famword{iLaaSak} & [iˌla{\longv}ˈsak] & presence of stress-carrying morpheme (imperative affix \emph{-ak})
    \end{tabular}
\end{table}


\section{Orthography}
\subsection{Formal writing style}

The formal writing conventions make use of small-caps letterforms to highlight roots. In addition, it uses the glottal stop character to indicate the glottal stop phoneme, using the capital glottal stop character \sqbrack{\bigglot} when the glottal stop is part of a root radical (for instance, in the word \textit{\bigglot a\bigglot a}) and the lowercase glottal stop character \sqbrack{\lilglot} otherwise (such as in the suffix \textit{-(e)}\lilglot).

\subsection{Informal writing style}

The informal writing conventions, also known as ``texting script", is the orthography used in the majority of day-to-day communication. Rather than using small-caps letterforms, it uses true capital letters for roots. It also uses \sqbrack{7} for the glottal stop, with no difference between capital and lowercase. While these differences could be considered less aesthetically pleasing, they result in an ASCII-compatible script, which makes this writing style far easier to use in most messaging apps and computer interfaces. Texting-style \lang{} also allows for several shorthand abbreviations that tend not to be used in more formal style.

\chapter{Word formation}
% Use the word formation section to describe 
% roots and patterns, then in each part of 
% speech chapter simply list the patterns and 
% what they do
\section{Roots}

The majority of lexical items are produced by applying morphological operation belonging to their respective morphological categories to abstract roots. These roots take the form of two \emph{radicals,} where each radical constitutes a non-zero number of consonants. 

Roots do not appear in isolation, and always combine with a pattern to form words.

There are several fossilized derivations within the roots whose meanings have been lost, and as such form discrete roots altogether. Compare \famword{L-S} \emph{`travel by foot'} and \famword{PL-S} \emph{`travel, habitation'}, or \famword{N-W} \emph{`death, murder'} and \famword{N-WK} \emph{`sacrifice, martyrdom'}.

\section{Primary derivation}

Primary derivation refers to the non-concatenative morphology of stems, from here on referred to as \textit{patterns}. These operations are for the most part not productive, and not all roots have a corresponding stem for every pattern. They may not stack, i.e. a stem may only be inflected by one pattern at a time.

\afterpage{%
\clearpage
\setcounter{numbertable}{0}
\begin{table}[p]
    \centering
    \begin{tabular}{@{}rllll@{}}
    & \textit{Pattern} & \textit{Meaning} & \textit{Example} & \\\refstepcounter{numbertable}
    \wtf{
        \rootpart{a}\rootpart & %
        Abstract noun & %
        \famword{JaB} & %
        good fortune \emph{(cf. \famword{JaBa} `good, fortunate')}
    }
    \wtf{
        \rootpart{ii}\rootpart & %
        Transitive verb & %
        \famword{FiiS} & %
        to give birth to \emph{(cf. \famword{FanaS} `person')}
    }
    \wtf{
        \rootpart{iya}\rootpart & %
        Unaccusative verb & %
        \famword{KiyaL} & %
        to be poured out \emph{(cf. \famword{KarLi} `water')}
    }
    \wtf{
        {i}\rootpart{aa}\rootpart & %
        Unergative verb & %
        \famword{iNaaM} & %
        to eat \emph{(cf. \famword{NiiM} `to eat (smth.)')}
    }
    \wtf{
        {\rootpart$_1$}i{\rootpart$_1$}iya{\rootpart$_2$} & %
        Causative of unaccusative & %
        \famword{KiKiyaL} & %
        to pour (smth.) out \emph{(cf. \famword{KiyaL} `to flow out')}
    }
    \wtf{
        aa\rootpart{i}\rootpart & %
        Causative of unergative & %
        \famword{aaNiM} & %
        to feed \textit{(cf. \famword{iNaaM} `to eat')}
    }
    \wtf{
        {\rootpart}{a}{\rootpart}{a} & %
        Attributive & %
        \famword{SaFRa} & %
        hot \emph{(cf. \famword{SaFR}`heat')}
    }
    \wtf{
        {u}\rootpart{i}\rootpart{i} & %
        Experiential & 
        \famword{uNiMi} & 
        hungry \emph{(cf. \famword{NaMa} `satisfying')}
    }
    \wtf{
        {\rootpart}{ana}{\rootpart}{} & %
        Person of X & %
        \famword{KanaJ}& %
        author \emph{(cf. \famword{KiiY} `to write (smth.)')}
    }
    \wtf{
        {\rootpart}{ur}{\rootpart}{i} & %
        Object & %
        \famword{NurMi}& %
        food \emph{(cf. \famword{iNaaM} `to eat')}
    }
    \wtf{
        {\rootpart}{ar}{\rootpart}{i} & %
        Liquid noun & %
        \famword{QarFi} & %
        coffee \emph{(cf. \famword{iQaaF} `to drink coffee')} 
    }
    \wtf{
        {i}\rootpart{u}\rootpart{a} & %
        Place of X & %
        \famword{iHuTa} & %
        night \emph{(cf. \famword{HaTa} `dark')}
    }
    \wtf{
        {m}\rootpart{i}\rootpart & %
        Tool/instrument & %
        \famword{mRiQ} & %
        weapon \emph{(cf. \famword{RaQ} `pain')}
    }
    \wtf{
        {in}\rootpart{i}\rootpart & %
        Diminutive & %
        \famword{inFiM} & %
        word \textit{(cf. \famword{FaM} `language')}
    }
    \wtf{
        \rootpart{u}\rootpart{i} & %
        Color/Visual Impression & %
        \famword{KuWi} & %
        green \emph{(cf., \famword{KajuWa} `leaf')}
    }
    \wtf{
        {\rootpart}uli{\rootpart} & %
        Body part & %
        \famword{BuliT}& %
        head \emph{(cf. \famword{iBaaT} `to understand')}
    }
    \wtf{
        \rootpart{u}\rootpart{u} & %
        Animal & %
        \famword{CuMPu} & %
        kangaroo \emph{(cf. \famword{iCaaMP} `to jump')}
    }
    \wtf{
        \rootpart{asi}\rootpart & %
        Long, slender object & %
        \famword{NasiRK} & %
        icicle \emph{(cf. \famword{NuRKi} `snowball')}
    }
    \wtf{
        \rootpart{aju}\rootpart{a} & %
        Flat object or surface & %
        \famword{DajuLa} & %
        mirror \emph{(cf. \famword{DiiL} `to stare at')}
    }
    \wtf{
        \rootpart{idi}\rootpart & %
        Loose, granular mass & %
        \famword{WidiW}& %
        sugar \emph{(cf. \famword{WaWa} `sweet')}
    }
    \wtf{
        {a}\rootpart{i}\rootpart{u} & %
        Closed/natural container & %
        \famword{aBiRDu} & %
        bird's nest \emph{(cf. \famword{BuRDu} `bird')} 
    }
    \wtf{
        \rootpart{imi}\rootpart{u} & %
        Open/unnatural container & %
        \famword{QimiFu} & %
        coffee mug \emph{(cf. \famword{aQiFu} `coffee pot')}
    }
    \wtf{
        {\rootpart$_1$}a{\rootpart$_2$}jala{\rootpart$_2$} &
        Upper-body clothing & 
        \famword{SaJejalaJ} & 
        pajamas \emph{(cf. \famword{iSaaJ} `to go to bed')}
    }
    \wtf{
        us\rootpart{}a\rootpart & 
        Lower-body clothing & 
        \famword{usRaT} & 
        skirt \emph{(cf. \famword{RaTa} `whole, round')}
    }
    \wtf{
        \rootpart u\rootpart aw & 
        Direction & 
        \famword{MuNTaw} & 
        east/west/towards the mountains \emph{(cf. \famword{iMuNTa})}
    }
    \wtf{
        \rootpart{uu}\rootpart & %
        People group, land of X people & %
        \famword{NuuRK} & %
        Nords, Norse, Norway \emph{(cf. \famword{NaRKa} `cold')}
    }
    \wtf{
        \rootpart{a}\rootpart{ia} & %
        Nationstate & %
        \famword{FRaNCia} & %
        France \emph{(cf. \famword{FRuuNC} `Franks')}
    }
    \end{tabular}
    \caption{Primary derivation patterns}
    \label{tab:primedevs}
\end{table}
\clearpage
}

\section{Secondary derivation}

Secondary derivation refers to the exclusively suffixing operations that may be applied to stems in addition to primary derivation, from here on referred to as \textit{suffixes}. Unlike with primary derivation, suffixes vary in productivity and may stack freely.

% Put a list here à la patterns list

\chapter{Verbs}

Verbs in \lang{} are highly inflected and very versatile

\section{Verbal patterns}

% Have sections here explaining each pattern

\subsection{\rootpart{}ii\rootpart{} - Transitive verb}

Transitive verbs take two arguments, one agentive and one patientive, and describe actions performed by the agent unto the patient. 

\subsection{\rootpart{}iya\rootpart{} - Unaccusative verb}

Unaccusative verbs are intransitive verbs formed using the \rootpart{}iya\rootpart{} pattern. They take a patientive argument.

\begin{table}[ht]
    \centering
    \begin{tabular}{ll}
        \famword{CiyaMP} & to bounce \\
        \famword{DiyaL} & to be noticed, appear \\
        \famword{FiyaL} & to get distracted \\
        \famword{HiyaR} & to burn \\
        \famword{KiyaW} & to grow \\
    \end{tabular}
    \caption{Examples of Unaccusative verbs}
\end{table}

A prototypical Unaccusative verb has one participant (or group of participants) who does not have control over the action or event; instead an external actor or force acts upon the participant without any explicit or implicit consent. For example, the verb \textit{\famword{CiyaMP}} `to bounce' refers to something or someone bouncing off a surface due to being thrown or dropped, not because it throws itself or jumps on its own. 

\subsection{i\rootpart{}aa\rootpart{} - Unergative verb}

Unergative verbs are intransitive verbs formed using the i\rootpart{}aa\rootpart{}. They take an agentive argument.

\begin{table}[ht]
    \centering
    \begin{tabular}{ll}
        \famword{iBaaRB} & to leave and return shortly \\
        \famword{iFaaL} & to stay alert \\
        \famword{iYaaN} & to relax \\
        \famword{iSaaFR} & to cook food \\
        \famword{iTaaT} & to participate \\
    \end{tabular}
    \caption{Examples of Unergative verbs}
\end{table}

A prototypical Unergative verb has one participant (or group of participants) who is performing an action of their own volition. The action can either involve the participant acting without influencing anyone else, e.g. \textit{\famword{iCaaJ}} `to hurry', or involve an implicit object or person, e.g. \textit{\famword{iKaaJ}} `to write'. 

\subsection{{\rootpart$_1$}i{\rootpart$_1$}iya{\rootpart$_2$} - Causative of unaccusative}

Unaccusative causative verbs are valency-increasing stems that derive their meaning from Unaccusative verbs. They take two arguments, an agentive causer who brings about the Unaccusative events that affect the original argument, now causee.

\subsection{aa\rootpart{}i\rootpart{} - Causative of unergative}

Unergatvie causative verbs are valency-increasing stems that derive their meaning from Unergative verbs. Like Unaccusative causative verbs, they introduce a causer with higher agency than the original arguments of the Unergative verbs.

\section{Verbal suffixes}

% Have sections here explaining each suffix

\subsection{-uru - `to be'}

The suffix \textit{-uru} derives copulative verbs from nominal or adjectival phrases. These convey the meaning of `to be X', and describe a given referent's state or characteristics. It behaves like a clitic, attaching to the entire phrase.

\pex
\a\begingl
\famword{PurLi}[apple]@
-uru[-\textsc{be}]
\glft `It's an apple.'
\endgl
\a\begingl
bus[\textsc{dem}]
\famword{MaLa}[ill]@
-uru[-\textsc{be}]
\glft `They're ill.'
\endgl
\a\begingl
pars[\textsc{dem}]
\famword{MaNTa}[big]
\famword{BuRKu}[dog]@
-uru[-\textsc{be}]
te[\textsc{atl}]
\glft `It was a big dog!'
\endgl
\a\begingl
\famword{\textrm{K}YIV-PLaS}[Kyiv]
\famword{WKRaNia}[the\_Ukraine]
fit[in]@
-uru[-\textsc{be}]
\glft `Kyiv is in the Ukraine.'
\endgl
\xe

\subsection{-ila - `to exist'}

The suffix \textit{-ila} derives existential verbs from nominal phrases. These predominantly convey the meaning of `there is X'. 

\subsection{-iri - `to make'}

The suffix \textit{-iri} derives verbs from nominal phrases describing the production of a particular thing.

\subsection{-ari - `to become, to cause to be'}

The suffix \textit{-ari} is a causative suffix.

\subsection{-inala - `to make X-er, to increase'}

The suffix \textit{-inala} 

\section{Animacy hierarchy and direct-inverse}

Unlike most languages, \lang{} makes use of neither case marking nor word order to indicate syntactic relationships in transitive sentences. Instead, the roles of the participants are decided based on their relative degree of animacy, where the more animate participant is considered the `subject' and the other the `object'. 

\begin{table}[ht]
    \centering
    \begin{tabular}{ll}
    0 & Natural Forces \\
    1 & Pronouns (1>2>3) \\
    2 & Speakers of \lang{} \\
    3 & Non-speakers of \lang{} \\
    4 & Higher-order animals (mammals, octopus, intelligent creatures) \\
    5 & Body parts, tools, any inanimate object used for acting upon something \\
    6 & Lower-order animals (insects, mollusks, fish, worms, etc.) \\
    7 & Plants, fungi, amoeba, etc. \\
    8 & Inanimate objects \\
    9 & Abstract concepts 
    \end{tabular}
    \caption{Animacy hierarchy in nominals}
    \label{tab:hierarchy}
\end{table}

If the object outranks the subject in animacy, then this must be marked with the \textit{-ibi} suffix on the verb.

\pex
\a\begingl
\glpreamble
\famword{nas CuSu FiiL.}
\endpreamble
nas[\textsc{1s}]
\famword{CuSu}[cat]
\famword{FiiL}[notice]
\glft `(as for me,) I saw the cat.'
\endgl
\a\begingl
\glpreamble
\famword{CuSu nas FiiL.}
\endpreamble
\famword{CuSu}[cat]
nas[\textsc{1s}]
\famword{FiiL}[notice]
\glft `As for the cat, I saw it.'
\endgl
\a\begingl
\glpreamble
\famword{CuSu nas FiiLibi.}
\endpreamble
\famword{Cusu}[cat]
nas[\textsc{1s}]
\famword{FiiL}[notice]@
-ibi[\textsc{-inv}]
\glft `As for the cat, it saw me.'
\endgl
\xe

\section{Verb stacking and connective verbs}

Verb phrases can contain several verbs, describing concurrent, subsequent, purposive, or consequential actions or states alongside the main verb. A verb phrase may contain at least one finite verb which counts as the main verb of the sentence that other constituents agree with. Other verbs are introduced with the connective suffix -\lilglot{}, and may either take the same arguments as the main verb or introduce their own. No syntactical or morphological distinction is made to differentiate the interactions a connective verb may have with the main verb, and is usually inferred through context. 

\ex
\begingl
\glpreamble
mi \famword{SiyaJe\lilglot{} hakli?}
\endpreamble
mi[\sc 2s]
SiyaJ[sleep]@
-\lilglot[\sc -cvb]
hak[continue]@
-li[\sc -q]
\glft `Are you still sleeping?'
\endgl
\xe

\ex
\begingl
\glpreamble
nas \famword{bu NuWu RiiQe\lilglot{} LaW daw iCaaN.}
\endpreamble
nas[\sc 1s]
bu[\sc dem]
\famword{NuWu}[possum]
\famword{RiiQ}[hit]@
-\lilglot[\sc cvb]
\famword{LaW}[top]
daw[towards]
\famword{iCaaN}[climb]
\glft `I'm climbing up to hit that possum.'
\endgl
\xe

\section{Auxiliary verbs}

The auxiliary verbs are a set of semantically sparse verbs that convey aspectual and modal information. These verbs are either the head verb or a connective (dependent) verb, but always semantically scope over the entire VP.

\subsection{\emph{hwii} - negative}

As the head verb, hwii scopes over the whole VP, negating all of its subordinate verbs. As a connective verb, it scopes leftward, negating any other connective verbs that precede it.

\subsection{\emph{usnak} - hortative}

from WeSiiN → usin + -ak → usnak

encodes a sort of imperative function so doesn't really take -ak suffix

\section{Subordinate clauses}

Full verb phrases may be nominalized and act as an argument of another predicate.

\subsection{Relative clauses}

Relative clauses are a type of subordinate clauses that describes a referent's states or actions. They are internally headed, always verb-final, and the relative determiner \emph{kun} is used to mark the head of the clause, i.e. the thing that is being described.

\ex
\begingl
\famword{FanaS}[person]
\famword{iLaaS}[walk]@
-tu[\textsc{-rel}]
\famword{SaJauru}[sleepy\textsc{:cop}]
\glft `The person who walked home was sleepy.'
\endgl
\xe

Clauses with a single argument do not require that the head is marked, as the argument is assumed to be the head by default. Still, the verb itself can be marked to describe the realization or performance of the action.

\ex
\begingl
\famword{inFiM}[children]
kun[\textsc{rel}]
\famword{iMaaW}[play]@
-tu[\textsc{-rel}]
naswi[\textsc{1p.ex}]
\famword{DiiL}[look]
\glft `We watched the playtime that the children were having'
\endgl
\xe

In high-valency clauses, \emph{kun} becomes more pertinent. The most agentive argument (subject) is considered to be the head of the phrase, but may still be marked for emphasis.

\pex
\a
\begingl
(kun)[\textsc{rel}]
\famword{FanaS}[person]
\famword{iFuSa}[house]
daw[to]
fit[in]
\famword{iLaaS}tu[walk\textsc{:rel}]
nas[\textsc{1s}]
\famword{FiiL}[see]
\glft `I saw the person who walked into the house.'
\endgl
\a
\begingl
\famword{FanaS}[person]
kun[\textsc{rel}]
\famword{iFuSa}[house]
daw[to]
fit[in]
\famword{iLaaS}tu[walk\textsc{:rel}]
nas[\textsc{1s}]
\famword{FiiL}[see]
\glft `I saw the house that the person walked into.'
\endgl
\a
\begingl
\famword{FanaS}[person]
\famword{iFuSa}[house]
daw[to]
fit[in]
kun[\textsc{rel}]
\famword{iLaaS}tu[walk\textsc{:rel}]
nas[\textsc{1s}]
\famword{FiiL}[see]
\glft `I saw how the person walked into the house.'
\endgl
\xe

An alternative to using a determiner is simply to topicalize a given constituent. Only noun phrases may be relativized through topicalization; the relative verb may not be periphrastically topicalized (i.e. left-dislocated), as this introduces major syntactical ambiguities.

Due to the syntactic constraints of certain secondary derivations, they cannot inflect relative NPs directly.

% \ex
% \begingl
% \glpreamble \ljudge{*} \famword{CuSu iFaaMtu-uru}
% \endpreamble
% \famword{CuSu}[cat]
% \famword{iFaaMtu}[jump\textsc{:rel}]
% kuns[\textsc{rel.pn}]@
% -uru[\textsc{-cop}]
% \glft `it's a talking cat.'
% \endgl
% \xe

\section{Causative constructions}

\lang{} has several different strategies when it comes to causative constructions, depending on the nature of the predicate in question. Some of these are morphological in nature, while others more periphrastic. 

\subsection{\textit{-ari} for nominal and adjectival predicates}

Simple nominal and adjectival predicates are turned into causatives using the translative suffix \textit{-ari}. If the predicate in question would be expressed with \textit{-uru} in its non-causative form, \textit{-ari} is likely appropriate for the causative.

\pex
\a
\begingl
\famword{QarFi}[coffee]
\famword{SaFRa}[hot]@
-uru[\textsc{-cop}]
\glft `The coffee is hot.'
\endgl
\a
\begingl
\famword{QarFi}[coffee]
nas[\textsc{1sg}]
\famword{SaFRa}[hot]@
-ari[-\textsc{transl}]
\glft `I heated up the coffee.'
\endgl
\xe

When used with only one argument, verbs ending in \textit{-ari} are assumed to have a null subject and the argument serving as the unaccusative object. This results in \textit{-ari} also serving as `to become' (the reason for its being glossed as `translative') as well as `to cause to be'.

\ex
\begingl
\famword{QarFi}[coffee]
\famword{SaFRa}[hot]@
-ari[\textsc{-transl}]
\glft `The coffee got hot.'
\endgl
\xe

\subsection{Valency-increasing verb patterns}

Which pattern is used to form the causative of a predicate depends largely on the nature of the intransitive form of that root. There are two different potentially valency-increasing patterns that can be used for verbs: the {\rootpart}ii{\rootpart} and the aa{\rootpart}i{\rootpart}. The exact effect of each of these valency-increasing operations depends on the individual root; their behavior can differ.

For verbs that would be agentive ambitransitives in English, such as `to eat', generally the behavior is rather straightforward: the {\rootpart}ii{\rootpart} form turns the verb into a straightfoward transitive, and the aa{\rootpart}i{\rootpart} form serves as a causative of the intransitive. 

\pex
\a
\begingl
nas[\textsc{1sg}]
\famword{iNaaM}[eat\textbackslash\textsc{intr}]
\glft `I was eating.'
\endgl
\a
\begingl
nas[\textsc{1sg}]
\famword{KurKi}[cookie]
\famword{NiiM}[eat\textbackslash\textsc{tr}]
\glft `I ate a cookie.'
\endgl
\a
\begingl
nas[\textsc{1sg}]
\famword{inMiM}[parent\_child\textbackslash\textsc{dim}]
\famword{aaNiM}[eat\textbackslash\textsc{caus}]
\glft `I fed my daughter.'
\endgl
\xe

It's worth noting that object of the transitive verb cannot be included as the object of the causative verb; the causative verb can still only have two arguments.

\ex
\ljudge{*}
\begingl
nas[\textsc{1sg}]
\famword{inMiM}[parent\_child\textbackslash\textsc{dim}]
\famword{KurKi}[cookie]
\famword{aaNiM}[eat\textbackslash\textsc{caus}]
\endgl
\xe

\noindent To express this notion, a periphrastic causative would be required.

Other types of verbal paradigms make this causative relationship less obvious and use these roots in other ways. For instance, for some roots the intransitive form is unaccusative or passive in nature. In these cases, the transitive form behaves as a causative:

\pex
\a
\begingl
nas[\textsc{1sg}]
wan[\textsc{poss}]
\famword{ManaM}[parent\_child]
\famword{NiyaW}[die\textbackslash\textsc{intr}]
\glft `My mother died.'
\endgl
\a
\begingl
nas[\textsc{1sg}]
\famword{ManaM}[parent\_child]
\famword{NiiW}[kill\textbackslash\textsc{tr}]
\glft `I killed my mother.'
\endgl
\xe

For these roots, the aa{\rootpart}i{\rootpart} form means the same thing as the {\rootpart}ii{\rootpart} form, but while the {\rootpart}ii{\rootpart} form implies a successfully completed action, the same implication is not present for the causative form.

\ex
\begingl
nas[\textsc{1sg}]
\famword{ManaM}[parent\_child]
\famword{NiNiyaW}[die\textbackslash\textsc{caus}]
\glft `I tried to kill my mother' (and she may or may not have died).
\endgl
\xe

\noindent For many of these roots, the intransitive is identical in meaning to a `passive' use of the transitive with an omitted subject; whether there is any noticeable difference between these depends on the verb.

\ex
\begingl
nas[\textsc{1sg}]
wan[\textsc{poss}]
\famword{ManaM}[parent\_child]
\famwordold{}{n}{ii}{w}{}[death\textbackslash\textsc{tr}]
\glft `My mother was killed.' (or `Someone killed my mother')
\endgl
\xe

% to-do
Unergative verbs

\subsection{Periphrastic causatives}

In addition to the morphological causatives above and their aforementioned limitations, \lang{} has a periphrastic causative that can scope over a wider variety of predicates. This periphrasis is expressed through a serial construction using the verb \textit{\textsc{w}\famwordold{e}{s}{ii}{n}{}} `to effect, to cause' followed by the description of the caused predicate. 

\ex
\begingl
nas[\textsc{1sg}]
\famword{WeSiiN}[bring\_about]
,[]
\famword{QarFi}[coffee]
mi[\textsc{2sg}]
\famword{KiiL}[drink]
\glft `I caused you to drink coffee.' (lit., `I brought it about, you drank coffee.')
\endgl
\xe

Insert stuff about causatives and directness here.

\section{Comparative constructions}

from-comparative, marks standard (to which is compared)

\pex
\a
\begingl
\famword{PuMu}[rabbit]
\famword{FanaS}[person]
fun[from]
\famword{MaNTa}[big]@
-uru[\textsc{-cop}]
\glft `The rabbit was bigger than a person.'
\endgl
\a
\begingl
\famword{TaN}[\textsc{top}/time]
nemi[\textsc{qual}/\textsc{du.in}]
buse[\textsc{std}/\textsc{dist:pn}]
fun[\textsc{mrk}/from]
\famword{JaL}[many\_things]@
-ila[/-have]
\glft `We have more time than them.'
\endgl
\xe

\chapter{Nominals}

The category of \textit{nominal} here refers to nouns, demonstrative and personal pronouns, and adjectives including numerals.

Nouns in \lang{}, in contrast with verbs, are grammatically very simple and have no inflection or agreement for any grammatical category, not even number. There are many times more nominal patterns covering different semantic categories.

\section{Nominal patterns}

% Have sections here explaining each pattern

\subsection{\rootpart{}a\rootpart{} - Abstract noun}

The Abstract noun pattern derives nouns with no physical referent, that is to say concepts and thoughts as opposed to corporeal entities or locations.

\subsection{\rootpart{}a\rootpart{}a - Attributive quality}

The Attributive pattern derives adjectives usually based on the most prominent characteristic or quality of a referent.

\subsection{u\rootpart{}i\rootpart{}i - Experential quality}

The Experiential pattern derives adjectives usually based on emotions and other internal impressions, e.g. hunger or the feeling of being useful.

\subsection{\rootpart{}ana\rootpart{} - Person}

The Person pattern derives nouns describing people or people-like referents.

\subsection{\rootpart{}ur\rootpart{}i - Object}

\subsection{\rootpart{}ar\rootpart{}i - Liquid}

\subsection{i\rootpart{}u\rootpart{}a - Place}

\subsection{m\rootpart{}i\rootpart{} - Tool or instrument}

\subsection{in\rootpart{}i\rootpart{} - Diminutive}

\subsection{\rootpart{}u\rootpart{}i - Color or visual impression}

\subsection{\rootpart{}uli\rootpart{} - Body part}

\subsection{\rootpart{}u\rootpart{}u - Animal}

\subsection{\rootpart{}asi\rootpart{} - Long slender object}

\subsection{\rootpart{}aju\rootpart{}a - Flat object or surface}

\subsection{\rootpart{}idi\rootpart{} - Loose, granular mass}

\subsection{a\rootpart{}i\rootpart{}u - Closed or natural container}

\subsection{\rootpart{}imi\rootpart{}u - Open or unnatural container}

\subsection{{\rootpart$_1$}a{\rootpart$_2$}jala{\rootpart$_2$} - Upper-body clothing}

\subsection{us\rootpart{}a\rootpart{} - Lower-body clothing}

\subsection{\rootpart{}u\rootpart{}aw - Direction}

\subsection{\rootpart{}uu\rootpart{} - People group}

\subsection{\rootpart{}a\rootpart{}ia - Nationstate}

\section{Nominal suffixes}

% Have sections here explaining each suffix

\subsection{-se - `the ... one'}

\subsection{-ana - Person}

\subsection{-irjan - Maker}

The suffix \textit{-irjan} is a shortening of the suffixes \textit{-iri + -ana}. 

\subsection{-arjan - Causer of change}

The suffix \textit{-arjan} is a shortening of the suffixes \textit{-ari + -ana}. 

\subsection{-ini - Diminutive}

\subsection{-lat - `measured in', `comprising'}

\subsection{-aki - `made/comprised of'}

\subsection{Gender}
Certain nouns may be derived to specific forms to convey the social gender of their referents. In phatic expressions (\emph{-ara} phrases), these gender suffixes are obligatory.

\begin{table}[ht]
    \centering
    \begin{tabular}{>{\em}ll}
    -un, -en, -um, -ine & Feminine gender \\
    -aj, -aa, -iy, -a & Masculine gender \\
    -uj, -aw & Explicitly non-binary \\
    -an, -en (-Ø) & Gender-neutral, agender \\
    \end{tabular}
    \caption{Common gender markers}
\end{table}

Unlike in many European languages where the linguistic term 'gender' refers to noun classes with some distinction aligned with the traditional binary gender distinction of male/female, the gender marking in \lang{} traditionally forms part of the gender expression of the individual in question, and is not assigned based on anything other than the individual's own feelings. It is entirely optional, but may be included for disambiguation or explicit declarations of gender identity. The table above contains only the four most commonly used categories of gender markers, and only a few variations taken from various lects.

When a noun is marked for gender, it is assumed that the speaker has prior knowledge about which marker to use for the referrent, and knows that the distinction is either desired or necessary in the context it is used in. This usually happens once on the first mention of a referrent, after which speakers may forego gender marking for the rest of the conversation. Some find it desirable to be referred to using their chosen gender marker on any mention of themselves, which usually happens in situations where their gender is significant to the context, but may also be requested for all mentions regardless of necessity.

While Nareland has traditionally had comparatively unbound and self-determinative concepts of gender identity and expression, they have been subjected to external pressures to conform to the gender binary practiced in the rest of Europe through Danish colonization. Due to the capital Masintulwa being predominantly culturally Danish, in the Masintulwa dialect, only the masculine and feminine gender markers are used, and only in the way you might see gender marking in other European languages (see e.g. English \textit{actor, husband, king} vs. \textit{actress, wife, queen}). However, in most non-urban areas of Nareland, and particularly among those striving to preserve Narish culture, it's still very common to see traditional Narish gender norms maintained. Gendered phatic expressions are prolific everywhere outside Masintulwa independent of other traditional practices.


\section{Personal pronouns}

\section{Determiners and demonstratives}

\section{Name determiners}

When a name is used referentially (that is, pointing out a particular entity named that), the name must be preceded by a name determiner. These determiners are only required when the name in question is serving a referential function, so they are not necessary when referring to the name itself as a concept (such as in `My name is ...' constructions) or in direct address.

\begin{table}[ht]
    \centering
    \begin{tabular}{ll}
        Feminine & fu \\
        Masculine & ha \\
        Nonbinary & daa (among others) \\
        Agender or unspecified & na 
    \end{tabular}
    \caption{Name determiners}
\end{table}

The male and female determiners, \textit{ha} and \textit{fu} respectively, are derived from former personal pronouns that have now been replaced by demonstratives in other contexts. More recently, \textit{na} has been innovated as a gender-neutral alternative (this new \textit{na} being unrelated to the former first-person determiner). Some non-binary alternatives are emerging, but none have gained widespread traction as of writing.

\pex
\a
\famword{ha \textrm{K}ARL-la fu \textrm{J}ANNE iLaaF.}\\
\textit{`Karl \& Janne are in love.'}
\a
\famword{bus \textrm{I}SKLIYANA baj iMaaH.}\\
\textit{`Their name is Iskliyana.'}
\a
\famword{\textrm{K}ARL, iWaaBak!}\\
\textit{`Karl, come back!'}
\xe

\chapter{Other Parts of Speech}

\section{Phatic expressions}

Phatic expressions in \lang{} are all in some way related to the nouns they are derived from, suggesting an emphasis on acknowledging the addressee's current or upcoming actions. The addressee may respond with the same expression back, even if it does not apply to the original speaker in any way, or respond in kind with a more suitable expression.

The obligatory gender marking is a means of expressing your gender identity in an unintrusive manner.\footnote{The real reason is that as Beth once ended a conversation with "sayonara", Kim noticed some coincidental similarities with the word \famword{SaJ} `sleep' and the affix -un to indicate feminine gender, with the -ara reanalyzed as a phatic/optative marker of sorts.} One always uses the gender marker that approximates ones gender identity (or at least intended gender presentation at the time), even if responding to someone else's phatic expression with a different gender marker. In contexts where the gender of the speaker is unclear or intentionally left unspecified, \textit{-an-} is used. More recently, \textit{-uj-} has been innovated as an explicitly nonbinary variant, though its use has been pretty much exclusively limited to the LGBTQ+ community.

In informal speech, most speakers shorten \textit{-unara} and \textit{-ajara} to \textit{-nar} and \textit{-jar}, respectively. While the full forms are generally used in formal contexts, the shorter forms are more likely to occur when speaking more casually. There are no generally accepted shortened variants for \textit{-anara}, which tends to not be used outside of formal contexts anyway, or \textit{-ujara}, whose use is limited anyway. 

In the Southernmost regions of Nareland, where contact with the Celts has always been strongest, most dialects have replaced \textit{-ara} with the Scottish Gaelic-inspired \textit{-maa} and \textit{-faa} replacing \textit{-jar} and \textit{-nar}, respectively. 
In urban areas in the East, particularly the capital, Danish influence has led to many speakers using \textit{kud X} (derived from Danish \textit{god}) in colloquial speech. 
In both places, however, the full form is generally still used for formal speech. A speaker of \lang{} from any region is likely to use \textit{\famword{FaSajara/FaSunara}} to greet someone at a job interview. 
With family and friends, a speaker from Masintulwa is likely to use \textit{\famword{kud FaS}}, a speaker from SOUTHERNCITY is likely to use \textit{\famword{FaSmaa/FaSfaa}}, and speakers from most other parts of the country are likely to use \textit{\famword{FaSjar/FaSnar}}.

\subsection*{Examples of Common Phatic Expressions}

\paragraph{\famword{FaSanara}} \textit{(from \famword{FaS} `life')} is a catch-all greeting, suitable for any time of day. It's generally only used in person.

\paragraph{\famword{SaJanara}} \textit{(from \famword{SaJ} `sleep')} is similar in use to "goodbye" or "good night". It is only used if the people in question intend to be separated for a period that includes a night of sleep before seeing each other again.

\paragraph{\famword{YaTanara}} \textit{(from \famword{YaT} `travel not of one's own power or volition')} is used to wish someone a pleasant trip where the person is not directly in control of their means of transportation (e.g. on public transport or as a passenger in a car). In contrast, if the person has direct control over their travel, e.g. by walking or driving a car, one would rather use \textbf{\famword{PLaSanara}} \textit{(from \famword{PLaS} `movement')} or \textbf{\famword{LaSanara}} \textit{(from \famword{LaS} `walking')}.

\chapter{Semantics and pragmatics}

\section{Postpositions and spatial relation}

When condisering the spatial relations between two referents, \lang{} does not consider the orientation of the referent unless it is the speaker themselves. The speaker is the deictic center unless reporting, in which case deictic center is transferred to that referent's perspective.

\section{Idiomatic expressions}

\famword{CuMPu CuMPuuru} = no shit, preaching to the choir

\part{Dictionary}

\include{dictionary/dictionary}

\part{Example Texts \& Translations}

%\include{examples/example_texts}

\end{document}

