\section{Phatic expressions}

Phatic expressions in \lang{} are all in some way related to the nouns they are derived from, suggesting an emphasis on acknowledging the addressee's current or upcoming actions. The addressee may respond with the same expression back, even if it does not apply to the original speaker in any way, or respond in kind with a more suitable expression.

The obligatory gender marking is a means of expressing your gender identity in an unintrusive manner.\footnote{The real reason is that as Beth once ended a conversation with "sayonara", Kim noticed some coincidental similarities with the word \famword{SaJ} `sleep' and the affix -un to indicate feminine gender, with the -ara reanalyzed as a phatic/optative marker of sorts.} One always uses the gender marker that approximates ones gender identity (or at least intended gender presentation at the time), even if responding to someone else's phatic expression with a different gender marker. In contexts where the gender of the speaker is unclear or intentionally left unspecified, \textit{-an-} is used. More recently, \textit{-uj-} has been innovated as an explicitly nonbinary variant, though its use has been pretty much exclusively limited to the LGBTQ+ community.

In informal speech, most speakers shorten \textit{-unara} and \textit{-ajara} to \textit{-nar} and \textit{-jar}, respectively. While the full forms are generally used in formal contexts, the shorter forms are more likely to occur when speaking more casually. There are no generally accepted shortened variants for \textit{-anara}, which tends to not be used outside of formal contexts anyway, or \textit{-ujara}, whose use is limited anyway. 

In the Southernmost regions of Nareland, where contact with the Celts has always been strongest, most dialects have replaced \textit{-ara} with the Scottish Gaelic-inspired \textit{-maa} and \textit{-faa} replacing \textit{-jar} and \textit{-nar}, respectively. 
In urban areas in the East, particularly the capital, Danish influence has led to many speakers using \textit{kud X} (derived from Danish \textit{god}) in colloquial speech. 
In both places, however, the full form is generally still used for formal speech. A speaker of \lang{} from any region is likely to use \textit{\famword{FaSajara/FaSunara}} to greet someone at a job interview. 
With family and friends, a speaker from Masintulwa is likely to use \textit{\famword{kud FaS}}, a speaker from SOUTHERNCITY is likely to use \textit{\famword{FaSmaa/FaSfaa}}, and speakers from most other parts of the country are likely to use \textit{\famword{FaSjar/FaSnar}}.

\subsection*{Examples of Common Phatic Expressions}

\paragraph{\famword{FaSanara}} \textit{(from \famword{FaS} `life')} is a catch-all greeting, suitable for any time of day. It's generally only used in person.

\paragraph{\famword{SaJanara}} \textit{(from \famword{SaJ} `sleep')} is similar in use to "goodbye" or "good night". It is only used if the people in question intend to be separated for a period that includes a night of sleep before seeing each other again.

\paragraph{\famword{YaTanara}} \textit{(from \famword{YaT} `travel not of one's own power or volition')} is used to wish someone a pleasant trip where the person is not directly in control of their means of transportation (e.g. on public transport or as a passenger in a car). In contrast, if the person has direct control over their travel, e.g. by walking or driving a car, one would rather use \textbf{\famword{PLaSanara}} \textit{(from \famword{PLaS} `movement')} or \textbf{\famword{LaSanara}} \textit{(from \famword{LaS} `walking')}.