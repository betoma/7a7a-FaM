\chapter{Verbs}

Verbs in \lang{} are highly inflected and very versatile

\section{Verbal patterns}

% Have sections here explaining each pattern

\subsection{\rootpart{}ii\rootpart{} - Transitive verb}

Transitive verbs take two arguments, one agentive and one patientive, and describe actions performed by the agent unto the patient. 

\subsection{\rootpart{}iya\rootpart{} - Unaccusative verb}

Unaccusative verbs are intransitive verbs formed using the \rootpart{}iya\rootpart{} pattern. They take a patientive argument.

% \begin{table}[ht]
%     \centering
%     \begin{tabular}{ll}
%         \famword{CiyaMP} & to bounce \\
%         \famword{DiyaL} & to be noticed, appear \\
%         \famword{FiyaL} & to get distracted \\
%         \famword{HiyaR} & to burn \\
%         \famword{KiyaW} & to grow \\
%     \end{tabular}
%     \caption{Examples of Unaccusative verbs}
% \end{table}

A prototypical Unaccusative verb has one participant (or group of participants) who does not have control over the action or event; instead an external actor or force acts upon the participant without any explicit or implicit consent. For example, the verb \textit{\famword{CiyaMP}} `to bounce' refers to something or someone bouncing off a surface due to being thrown or dropped, not because it throws itself or jumps on its own. 

\subsection{i\rootpart{}aa\rootpart{} - Unergative verb}

Unergative verbs are intransitive verbs formed using the i\rootpart{}aa\rootpart{}. They take an agentive argument.

% \begin{table}[ht]
%     \centering
%     \begin{tabular}{ll}
%         \famword{iBaaRB} & to leave and return shortly \\
%         \famword{iFaaL} & to stay alert \\
%         \famword{iYaaN} & to relax \\
%         \famword{iSaaFR} & to cook food \\
%         \famword{iTaaT} & to participate \\
%     \end{tabular}
%     \caption{Examples of Unergative verbs}
% \end{table}

A prototypical Unergative verb has one participant (or group of participants) who is performing an action of their own volition. The action can either involve the participant acting without influencing anyone else, e.g. \textit{\famword{iCaaJ}} `to hurry', or involve an implicit object or person, e.g. \textit{\famword{iKaaJ}} `to write'. 

\subsection{{\rootpart$_1$}i{\rootpart$_1$}iya{\rootpart$_2$} - Causative of unaccusative}

Unaccusative causative verbs are valency-increasing stems that derive their meaning from Unaccusative verbs. They take two arguments, an agentive causer who brings about the Unaccusative events that affect the original argument, now causee.

\subsection{aa\rootpart{}i\rootpart{} - Causative of unergative}

Unergatvie causative verbs are valency-increasing stems that derive their meaning from Unergative verbs. Like Unaccusative causative verbs, they introduce a causer with higher agency than the original arguments of the Unergative verbs.

\section{Verbal suffixes}

% Have sections here explaining each suffix

\subsection{-uru - `to be'}

The suffix \textit{-uru} derives copulative verbs from nominal or adjectival phrases. These convey the meaning of `to be X', and describe a given referent's state or characteristics. It behaves like a clitic, attaching to the entire phrase.

\pex
\a\begingl
\famword{PurLi}[apple]@
-uru[-\textsc{be}]
\glft `It's an apple.'
\endgl
\a\begingl
bus[\textsc{dem}]
\famword{MaLa}[ill]@
-uru[-\textsc{be}]
\glft `They're ill.'
\endgl
\a\begingl
pars[\textsc{dem}]
\famword{MaNTa}[big]
\famword{BuRKu}[dog]@
-uru[-\textsc{be}]
te[\textsc{atl}]
\glft `It was a big dog!'
\endgl
\a\begingl
\famword{\textrm{K}YIV-PLaS}[Kyiv]
\famword{WKRaNia}[the\_Ukraine]
fit[in]@
-uru[-\textsc{be}]
\glft `Kyiv is in the Ukraine.'
\endgl
\xe

\subsection{-ila - `to exist'}

The suffix \textit{-ila} derives existential verbs from nominal phrases. These predominantly convey the meaning of `there is X'. 

\subsection{-iri - `to make'}

The suffix \textit{-iri} derives verbs from nominal phrases describing the production of a particular thing.

\subsection{-ari - `to become, to cause to be'}

The suffix \textit{-ari} is a causative suffix.

\subsection{-inala - `to make X-er, to increase'}

The suffix \textit{-inala} 

\section{Animacy hierarchy and direct-inverse}

Unlike most languages, \lang{} makes use of neither case marking nor word order to indicate syntactic relationships in transitive sentences. Instead, the roles of the participants are decided based on their relative degree of animacy, where the more animate participant is considered the `subject' and the other the `object'. 

% \begin{table}[ht]
%     \centering
%     \begin{tabular}{ll}
%     0 & Natural Forces \\
%     1 & Pronouns (1>2>3) \\
%     2 & Speakers of \lang{} \\
%     3 & Non-speakers of \lang{} \\
%     4 & Higher-order animals (mammals, octopus, intelligent creatures) \\
%     5 & Body parts, tools, any inanimate object used for acting upon something \\
%     6 & Lower-order animals (insects, mollusks, fish, worms, etc.) \\
%     7 & Plants, fungi, amoeba, etc. \\
%     8 & Inanimate objects \\
%     9 & Abstract concepts 
%     \end{tabular}
%     \caption{Animacy hierarchy in nominals}
%     \label{tab:hierarchy}
% \end{table}

If the object outranks the subject in animacy, then this must be marked with the \textit{-ibi} suffix on the verb.

\pex
\a\begingl
\glpreamble
\famword{nas CuSu FiiL.}
\endpreamble
nas[\textsc{1s}]
\famword{CuSu}[cat]
\famword{FiiL}[notice]
\glft `(as for me,) I saw the cat.'
\endgl
\a\begingl
\glpreamble
\famword{CuSu nas FiiL.}
\endpreamble
\famword{CuSu}[cat]
nas[\textsc{1s}]
\famword{FiiL}[notice]
\glft `As for the cat, I saw it.'
\endgl
\a\begingl
\glpreamble
\famword{CuSu nas FiiLibi.}
\endpreamble
\famword{Cusu}[cat]
nas[\textsc{1s}]
\famword{FiiL}[notice]@
-ibi[\textsc{-inv}]
\glft `As for the cat, it saw me.'
\endgl
\xe

\section{Verb stacking and connective verbs}

Verb phrases can contain several verbs, describing concurrent, subsequent, purposive, or consequential actions or states alongside the main verb. A verb phrase may contain at least one finite verb which counts as the main verb of the sentence that other constituents agree with. Other verbs are introduced with the connective suffix -\lilglot{}, and may either take the same arguments as the main verb or introduce their own. No syntactical or morphological distinction is made to differentiate the interactions a connective verb may have with the main verb, and is usually inferred through context. 

\ex
\begingl
\glpreamble
mi \famword{SiyaJe\lilglot{} hakli?}
\endpreamble
mi[\sc 2s]
SiyaJ[sleep]@
-\lilglot[\sc -cvb]
hak[continue]@
-li[\sc -q]
\glft `Are you still sleeping?'
\endgl
\xe

\ex
\begingl
\glpreamble
nas \famword{bu NuWu RiiQe\lilglot{} LaW daw iCaaN.}
\endpreamble
nas[\sc 1s]
bu[\sc dem]
\famword{NuWu}[possum]
\famword{RiiQ}[hit]@
-\lilglot[\sc cvb]
\famword{LaW}[top]
daw[towards]
\famword{iCaaN}[climb]
\glft `I'm climbing up to hit that possum.'
\endgl
\xe

\section{Auxiliary verbs}

The auxiliary verbs are a set of semantically sparse verbs that convey aspectual and modal information. These verbs are either the head verb or a connective (dependent) verb, but always semantically scope over the entire VP.

\subsection{\emph{hwii} - negative}

As the head verb, hwii scopes over the whole VP, negating all of its subordinate verbs. As a connective verb, it scopes leftward, negating any other connective verbs that precede it.

\subsection{\emph{usnak} - hortative}

from WeSiiN → usin + -ak → usnak

encodes a sort of imperative function so doesn't really take -ak suffix

\section{Subordinate clauses}

Full verb phrases may be nominalized and act as an argument of another predicate.

\subsection{Relative clauses}

Relative clauses are a type of subordinate clauses that describes a referent's states or actions. They are internally headed, always verb-final, and the relative determiner \emph{kun} is used to mark the head of the clause, i.e. the thing that is being described.

\ex
\begingl
\famword{FanaS}[person]
\famword{iLaaS}[walk]@
-tu[\textsc{-rel}]
\famword{SaJauru}[sleepy\textsc{:cop}]
\glft `The person who walked home was sleepy.'
\endgl
\xe

Clauses with a single argument do not require that the head is marked, as the argument is assumed to be the head by default. Still, the verb itself can be marked to describe the realization or performance of the action.

\ex
\begingl
\famword{inFiM}[children]
kun[\textsc{rel}]
\famword{iMaaW}[play]@
-tu[\textsc{-rel}]
naswi[\textsc{1p.ex}]
\famword{DiiL}[look]
\glft `We watched the playtime that the children were having'
\endgl
\xe

In high-valency clauses, \emph{kun} becomes more pertinent. The most agentive argument (subject) is considered to be the head of the phrase, but may still be marked for emphasis.

\pex
\a
\begingl
(kun)[\textsc{rel}]
\famword{FanaS}[person]
\famword{iFuSa}[house]
daw[to]
fit[in]
\famword{iLaaS}tu[walk\textsc{:rel}]
nas[\textsc{1s}]
\famword{FiiL}[see]
\glft `I saw the person who walked into the house.'
\endgl
\a
\begingl
\famword{FanaS}[person]
kun[\textsc{rel}]
\famword{iFuSa}[house]
daw[to]
fit[in]
\famword{iLaaS}tu[walk\textsc{:rel}]
nas[\textsc{1s}]
\famword{FiiL}[see]
\glft `I saw the house that the person walked into.'
\endgl
\a
\begingl
\famword{FanaS}[person]
\famword{iFuSa}[house]
daw[to]
fit[in]
kun[\textsc{rel}]
\famword{iLaaS}tu[walk\textsc{:rel}]
nas[\textsc{1s}]
\famword{FiiL}[see]
\glft `I saw how the person walked into the house.'
\endgl
\xe

An alternative to using a determiner is simply to topicalize a given constituent. Only noun phrases may be relativized through topicalization; the relative verb may not be periphrastically topicalized (i.e. left-dislocated), as this introduces major syntactical ambiguities.

Due to the syntactic constraints of certain secondary derivations, they cannot inflect relative NPs directly.

% \ex
% \begingl
% \glpreamble \ljudge{*} \famword{CuSu iFaaMtu-uru}
% \endpreamble
% \famword{CuSu}[cat]
% \famword{iFaaMtu}[jump\textsc{:rel}]
% kuns[\textsc{rel.pn}]@
% -uru[\textsc{-cop}]
% \glft `it's a talking cat.'
% \endgl
% \xe

\section{Causative constructions}

\lang{} has several different strategies when it comes to causative constructions, depending on the nature of the predicate in question. Some of these are morphological in nature, while others more periphrastic. 

\subsection{\textit{-ari} for nominal and adjectival predicates}

Simple nominal and adjectival predicates are turned into causatives using the translative suffix \textit{-ari}. If the predicate in question would be expressed with \textit{-uru} in its non-causative form, \textit{-ari} is likely appropriate for the causative.

\pex
\a
\begingl
\famword{QarFi}[coffee]
\famword{SaFRa}[hot]@
-uru[\textsc{-cop}]
\glft `The coffee is hot.'
\endgl
\a
\begingl
\famword{QarFi}[coffee]
nas[\textsc{1sg}]
\famword{SaFRa}[hot]@
-ari[-\textsc{transl}]
\glft `I heated up the coffee.'
\endgl
\xe

When used with only one argument, verbs ending in \textit{-ari} are assumed to have a null subject and the argument serving as the unaccusative object. This results in \textit{-ari} also serving as `to become' (the reason for its being glossed as `translative') as well as `to cause to be'.

\ex
\begingl
\famword{QarFi}[coffee]
\famword{SaFRa}[hot]@
-ari[\textsc{-transl}]
\glft `The coffee got hot.'
\endgl
\xe

\subsection{Valency-increasing verb patterns}

Which pattern is used to form the causative of a predicate depends largely on the nature of the intransitive form of that root. There are two different potentially valency-increasing patterns that can be used for verbs: the {\rootpart}ii{\rootpart} and the aa{\rootpart}i{\rootpart}. The exact effect of each of these valency-increasing operations depends on the individual root; their behavior can differ.

For verbs that would be agentive ambitransitives in English, such as `to eat', generally the behavior is rather straightforward: the {\rootpart}ii{\rootpart} form turns the verb into a straightfoward transitive, and the aa{\rootpart}i{\rootpart} form serves as a causative of the intransitive. 

\pex
\a
\begingl
nas[\textsc{1sg}]
\famword{iNaaM}[eat\textbackslash\textsc{intr}]
\glft `I was eating.'
\endgl
\a
\begingl
nas[\textsc{1sg}]
\famword{KurKi}[cookie]
\famword{NiiM}[eat\textbackslash\textsc{tr}]
\glft `I ate a cookie.'
\endgl
\a
\begingl
nas[\textsc{1sg}]
\famword{inMiM}[parent\_child\textbackslash\textsc{dim}]
\famword{aaNiM}[eat\textbackslash\textsc{caus}]
\glft `I fed my daughter.'
\endgl
\xe

It's worth noting that object of the transitive verb cannot be included as the object of the causative verb; the causative verb can still only have two arguments.

\ex
\ljudge{*}
\begingl
nas[\textsc{1sg}]
\famword{inMiM}[parent\_child\textbackslash\textsc{dim}]
\famword{KurKi}[cookie]
\famword{aaNiM}[eat\textbackslash\textsc{caus}]
\endgl
\xe

\noindent To express this notion, a periphrastic causative would be required.

Other types of verbal paradigms make this causative relationship less obvious and use these roots in other ways. For instance, for some roots the intransitive form is unaccusative or passive in nature. In these cases, the transitive form behaves as a causative:

\pex
\a
\begingl
nas[\textsc{1sg}]
wan[\textsc{poss}]
\famword{ManaM}[parent\_child]
\famword{NiyaW}[die\textbackslash\textsc{intr}]
\glft `My mother died.'
\endgl
\a
\begingl
nas[\textsc{1sg}]
\famword{ManaM}[parent\_child]
\famword{NiiW}[kill\textbackslash\textsc{tr}]
\glft `I killed my mother.'
\endgl
\xe

For these roots, the aa{\rootpart}i{\rootpart} form means the same thing as the {\rootpart}ii{\rootpart} form, but while the {\rootpart}ii{\rootpart} form implies a successfully completed action, the same implication is not present for the causative form.

\ex
\begingl
nas[\textsc{1sg}]
\famword{ManaM}[parent\_child]
\famword{NiNiyaW}[die\textbackslash\textsc{caus}]
\glft `I tried to kill my mother' (and she may or may not have died).
\endgl
\xe

\noindent For many of these roots, the intransitive is identical in meaning to a `passive' use of the transitive with an omitted subject; whether there is any noticeable difference between these depends on the verb.

\ex
\begingl
nas[\textsc{1sg}]
wan[\textsc{poss}]
\famword{ManaM}[parent\_child]
\famwordold{}{n}{ii}{w}{}[death\textbackslash\textsc{tr}]
\glft `My mother was killed.' (or `Someone killed my mother')
\endgl
\xe

% to-do
Unergative verbs

\subsection{Periphrastic causatives}

In addition to the morphological causatives above and their aforementioned limitations, \lang{} has a periphrastic causative that can scope over a wider variety of predicates. This periphrasis is expressed through a serial construction using the verb \textit{\textsc{w}\famwordold{e}{s}{ii}{n}{}} `to effect, to cause' followed by the description of the caused predicate. 

\ex
\begingl
nas[\textsc{1sg}]
\famword{WeSiiN}[bring\_about]
,[]
\famword{QarFi}[coffee]
mi[\textsc{2sg}]
\famword{KiiL}[drink]
\glft `I caused you to drink coffee.' (lit., `I brought it about, you drank coffee.')
\endgl
\xe

Insert stuff about causatives and directness here.

\section{Comparative constructions}

from-comparative, marks standard (to which is compared)

\pex
\a
\begingl
\famword{PuMu}[rabbit]
\famword{FanaS}[person]
fun[from]
\famword{MaNTa}[big]@
-uru[\textsc{-cop}]
\glft `The rabbit was bigger than a person.'
\endgl
\a
\begingl
\famword{TaN}[\textsc{top}/time]
nemi[\textsc{qual}/\textsc{du.in}]
buse[\textsc{std}/\textsc{dist:pn}]
fun[\textsc{mrk}/from]
\famword{JaL}[many\_things]@
-ila[/-have]
\glft `We have more time than them.'
\endgl
\xe