\chapter{Nominals}

The category of \textit{nominal} here refers to nouns, demonstrative and personal pronouns, and adjectives including numerals.

Nouns in \lang{}, in contrast with verbs, are grammatically very simple and have no inflection or agreement for any grammatical category, not even number. There are many times more nominal patterns covering different semantic categories.

\section{Nominal patterns}

% Have sections here explaining each pattern

\subsection{\rootpart{}a\rootpart{} - Abstract noun}

The Abstract noun pattern derives nouns with no physical referent, that is to say concepts and thoughts as opposed to corporeal entities or locations.

\subsection{\rootpart{}a\rootpart{}a - Attributive quality}

The Attributive pattern derives adjectives usually based on the most prominent characteristic or quality of a referent.

\subsection{u\rootpart{}i\rootpart{}i - Experential quality}

The Experiential pattern derives adjectives usually based on emotions and other internal impressions, e.g. hunger or the feeling of being useful.

\subsection{\rootpart{}ana\rootpart{} - Person}

The Person pattern derives nouns describing people or people-like referents.

\subsection{\rootpart{}ur\rootpart{}i - Object}

\subsection{\rootpart{}ar\rootpart{}i - Liquid}

\subsection{i\rootpart{}u\rootpart{}a - Place}

\subsection{m\rootpart{}i\rootpart{} - Tool or instrument}

\subsection{in\rootpart{}i\rootpart{} - Diminutive}

\subsection{\rootpart{}u\rootpart{}i - Color or visual impression}

\subsection{\rootpart{}uli\rootpart{} - Body part}

\subsection{\rootpart{}u\rootpart{}u - Animal}

\subsection{\rootpart{}asi\rootpart{} - Long slender object}

\subsection{\rootpart{}aju\rootpart{}a - Flat object or surface}

\subsection{\rootpart{}idi\rootpart{} - Loose, granular mass}

\subsection{a\rootpart{}i\rootpart{}u - Closed or natural container}

\subsection{\rootpart{}imi\rootpart{}u - Open or unnatural container}

\subsection{{\rootpart$_1$}a{\rootpart$_2$}jala{\rootpart$_2$} - Upper-body clothing}

\subsection{us\rootpart{}a\rootpart{} - Lower-body clothing}

\subsection{\rootpart{}u\rootpart{}aw - Direction}

\subsection{\rootpart{}uu\rootpart{} - People group}

\subsection{\rootpart{}a\rootpart{}ia - Nationstate}

\section{Nominal suffixes}

% Have sections here explaining each suffix

\subsection{-se - `the ... one'}

\subsection{-ana - Person}

\subsection{-irjan - Maker}

The suffix \textit{-irjan} is a shortening of the suffixes \textit{-iri + -ana}. 

\subsection{-arjan - Causer of change}

The suffix \textit{-arjan} is a shortening of the suffixes \textit{-ari + -ana}. 

\subsection{-ini - Diminutive}

\subsection{-lat - `measured in', `comprising'}

\subsection{-aki - `made/comprised of'}

\subsection{Gender}
Certain nouns may be derived to specific forms to convey the social gender of their referents. In phatic expressions (\emph{-ara} phrases), these gender suffixes are obligatory.

% \begin{table}[ht]
%     \centering
%     \begin{tabular}{>{\em}ll}
%     -un, -en, -um, -ine & Feminine gender \\
%     -aj, -aa, -iy, -a & Masculine gender \\
%     -uj, -aw & Explicitly non-binary \\
%     -an, -en (-Ø) & Gender-neutral, agender \\
%     \end{tabular}
%     \caption{Common gender markers}
% \end{table}

Unlike in many European languages where the linguistic term 'gender' refers to noun classes with some distinction aligned with the traditional binary gender distinction of male/female, the gender marking in \lang{} traditionally forms part of the gender expression of the individual in question, and is not assigned based on anything other than the individual's own feelings. It is entirely optional, but may be included for disambiguation or explicit declarations of gender identity. The table above contains only the four most commonly used categories of gender markers, and only a few variations taken from various lects.

When a noun is marked for gender, it is assumed that the speaker has prior knowledge about which marker to use for the referrent, and knows that the distinction is either desired or necessary in the context it is used in. This usually happens once on the first mention of a referrent, after which speakers may forego gender marking for the rest of the conversation. Some find it desirable to be referred to using their chosen gender marker on any mention of themselves, which usually happens in situations where their gender is significant to the context, but may also be requested for all mentions regardless of necessity.

While Nareland has traditionally had comparatively unbound and self-determinative concepts of gender identity and expression, they have been subjected to external pressures to conform to the gender binary practiced in the rest of Europe through Danish colonization. Due to the capital Masintulwa being predominantly culturally Danish, in the Masintulwa dialect, only the masculine and feminine gender markers are used, and only in the way you might see gender marking in other European languages (see e.g. English \textit{actor, husband, king} vs. \textit{actress, wife, queen}). However, in most non-urban areas of Nareland, and particularly among those striving to preserve Narish culture, it's still very common to see traditional Narish gender norms maintained. Gendered phatic expressions are prolific everywhere outside Masintulwa independent of other traditional practices.


\section{Personal pronouns}

\section{Determiners and demonstratives}

\section{Name determiners}

When a name is used referentially (that is, pointing out a particular entity named that), the name must be preceded by a name determiner. These determiners are only required when the name in question is serving a referential function, so they are not necessary when referring to the name itself as a concept (such as in `My name is ...' constructions) or in direct address.

% \begin{table}[ht]
%     \centering
%     \begin{tabular}{ll}
%         Feminine & fu \\
%         Masculine & ha \\
%         Nonbinary & daa (among others) \\
%         Agender or unspecified & na 
%     \end{tabular}
%     \caption{Name determiners}
% \end{table}

The male and female determiners, \textit{ha} and \textit{fu} respectively, are derived from former personal pronouns that have now been replaced by demonstratives in other contexts. More recently, \textit{na} has been innovated as a gender-neutral alternative (this new \textit{na} being unrelated to the former first-person determiner). Some non-binary alternatives are emerging, but none have gained widespread traction as of writing.

\pex
\a
\famword{ha \textrm{K}ARL-la fu \textrm{J}ANNE iLaaF.}\\
\textit{`Karl \& Janne are in love.'}
\a
\famword{bus \textrm{I}SKLIYANA baj iMaaH.}\\
\textit{`Their name is Iskliyana.'}
\a
\famword{\textrm{K}ARL, iWaaBak!}\\
\textit{`Karl, come back!'}
\xe